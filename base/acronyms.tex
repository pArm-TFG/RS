\begin{acronym}
    \acro{SDK}{\textit{Software Development Kit}}
    \acro{ROS}{\textit{Robot Operating System}}
    \acro{API}{\textit{Application Programming Interface}}
    \acro{SW}{\textit{software}}
    \acro{HW}{\textit{hardware}}
    \acro{pArm}{\textit{Printed -- Arm}}
    \acro{USB}{\textit{Universal Serial Bus}}
    \acro{ODS}{Objetivos de Desarrollo Sostenible}
    \acro{OS}{\textit{Open--Source}}
    \acro{OH}{\textit{Open--Hardware}}
    \acro{S1}{Sistema 1 -- ordenador}
    \acro{S2}{Sistema 2 -- \pArm{}}
    \acro{GUI}{\textit{Graphical User Interface}}
    \acro{GTK}{\textit{GIMP Toolkit}}
    \acro{SoC}{\textit{System On Chip}}
    \acro{PWM}{\textit{Pulse--Width Modulation}}
    \acro{GPIO}{\textit{General Purpose Input/Output}}
    \acro{UART}{\textit{Universal Asynchronous Receiver--Transmitter}}
    \acro{RAM}{\textit{Random Access Memory}}
    \acro{ADC}{\textit{Analog--Digital Conversor}}
\end{acronym}

\begin{itemize}
    \item \ac{SDK} -- colección de herramientas \ac{SW} disponibles para instalar en un único paquete.
    \item \ac{ROS} -- conjunto de librerías \ac{SW} que ayudan a construir aplicaciones para robots.
    \item \textit{Firmware} -- \ac{SW} programado que especifica el orden de ejecución del sistema.
    \item \ac{GUI} -- siglas que significan ``Interfaz Gráfica de Usuario'' (en castellano).
    \item \ac{GTK} -- biblioteca de componentes gráficos multiplataforma para desarrollar interfaces gráficas de usuario.
    \item \ac{SoC} -- tecnología de fabricación que integra todos o gran parte de los módulos en un circuito integrado. Un ejemplo muy común es la placa base de un teléfono móvil, la cual es un \ac{SoC} que integra todos los componentes (antenas, conversores, sensores, etc.).
    \item \ac{PWM} -- señal en la cual se modifica el ciclo de trabajo de una señal periódica, para transmitir información por un canal de comunicaciones o para controlar la cantidad de energía que se envía a una carga.
    \item \ac{GPIO} -- pin genérico cuyo comportamiento puede ser controlado en tiempo de ejecución.
    \item \ac{UART} -- estándar de comunicación dúplex simultáneos.
    \item Dúplex -- término que define a un sistema que es capaz de mantener una comunicación bidireccional, enviando y recibiendo mensajes de forma simultánea.
    \item Widget -- la parte de una GUI (interfaz gráfica de usuario) que permite al usuario interconectar con la aplicación y el sistema operativo.
    \item \ac{RAM} -- memoria principal del ordenador, donde se guardan programas y datos, sobre la que se pueden efectuar operaciones de lectura y escritura.
    \item \textit{Deep--Sleep} -- estado de un microcontrolador por el cual consume muy poca cantidad de energía.
    \item \textit{bit} -- unidad mínima de información de una computadora.
\end{itemize}
