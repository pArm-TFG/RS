Tanto para el \ac{SW} como para el \ac{HW}, se busca que ambos cumplan las siguientes premisas:

\begin{enumerate}
    \item El sistema al completo ha de ser seguro, en el rango del brazo robótico. Esto es, no se permitirá a \ac{S2} realizar movimientos que puedan perjudicar la estructura del mismo de forma irremediable. A su vez, el sistema \ac{S2} deberá tener en cuenta posibles fallos en las órdenes de \ac{S1} y comprobar así que la secuencia de órdenes es correcta y no contiene posiciones inseguras.
    \item Teniendo en cuenta lo desarrollado en el punto de ``Descripción general'' (\ref{Descripción general}) y lo mencionado en la ``Introducción'' (\ref{ch:intro}), es importante que el sistema sea mantenible. Esto se traduce en que, por una parte, se pueda actualizar para corregir problemas que se han encontrado una vez se ha desplegado el sistema; y que la sustitución de piezas o elementos del mismo resulte accesible y barato.
    \item Finalmente, dado que el \pArm{} está impreso en 3D, se busca que sea portable en lo referente a que pueda ser fácilmente transportado de un lugar a otro. Esto se traducirá en un bajo peso y que el área ocupada por el mismo sea también baja.
\end{enumerate}