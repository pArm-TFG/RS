\subsubsection{Interfaz con el usuario}
\label{sec:ui_reqs}
\ac{S1} dispondrá de una interfaz de usuario que deberá seguir el modelo propuesto en la figura \ref{fig:ui_design}. Dicha interfaz habrá de contar con los siguientes elementos:
\begin{itemize}
    \item Distintos controladores gráficos que permitan establecer la posición final del \textit{end--effector} bien mediante coordenadas articulares o bien mediante coordenadas angulares.
    \item Un actuador para poder escoger entre las alternativas mencionadas en el punto anterior.
    \item Un actuador para confirmar que se quiere mandar el movimiento al brazo robótico.
    \item Un actuador para detener un movimiento en ejecución del brazo.
\end{itemize}

Teniendo en cuenta el diseño propuesto en la figura \ref{fig:ui_design}, los componentes anteriores estarían
representados por:

\begin{itemize}
    \item Tres \textit{slider}s los cuales establecerán los valores para los ángulos 
    $\left\{\theta_1, \theta_2, \theta_3\right\}$, si se está trabajando en el modo de coordenadas
    angulares; o los valores de los puntos $\left\{x, y, z\right\}$, si se está trabajando en el modo
    de coordenadas cartesianas.
    \item Un botón desplegable con múltiples opciones que permitiría escoger entre los dos modos
    mencionados en el punto anterior.
    \item Un botón que permita confirmar los cambios en las coordenadas/ángulos antes de enviar
    definitivamente el movimiento al robot.
    \item Una barra de progreso la cual permite conocer una estimación de cuánto lleva el robot hecho
    del movimiento final previsto.
    \item Dos pequeñas ventanas que informan sobre la posición del brazo final una vez se han
    cambiado los valores de las coordenadas/ángulos. Dichas ventanas muestra una vista cenital del
    brazo, que indica cómo se mueve en el eje $Y$, y una vista de perfil del mismo, que indica cómo
    se mueve en el eje $XZ$.
    \item Una pequeña ventana que muestra \textit{logs} relevantes respecto a la situación tanto
    de \ac{S1} como de \ac{S2}.
\end{itemize}