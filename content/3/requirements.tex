\subsubsection{Permitir generar movimientos}
\subsubsection*{ -- Mediante el ángulo de cada uno de los motores}
El sistema \ac{S2} cuenta con tres motores los cuales se encargan del movimiento de cada una de las partes del
brazo. Se permitirá establecer individualmente cada ángulo $\left\{\theta_0, \theta_2, \theta_3\right\}$ 
y mover así el brazo a una posición final $\left\{x', y', z'\right\}$.

Este movimiento, siguiendo la maqueta definida en la figura \ref{fig:ui_design}, se realizará interactuando
con \textit{sliders}.

\subsubsection*{ -- Mediante las coordenadas cartesianas del punto final}
Se permitirá también el movimiento a un punto $\left\{x, y, z\right\}$ directamente, para lo que se obtendrán
los ángulos $\left\{\theta_0, \theta_2, \theta_3\right\}$ que permiten alcanzar dicha posición.

Al igual que en el caso anterior, se podrá definir cada punto independientemente.

Este movimiento, siguiendo la maqueta definida en la figura \ref{fig:ui_design}, se realizará interactuando
con \textit{sliders}.

\subsubsection*{ -- Selección del modo de funcionamiento del brazo}
Dado que, como se ha mencionado en los puntos anteriores, hay dos maneras de hacer que el brazo se pueda
mover, habrá de existir algún tipo de actuador en la interfaz de usuario que permita escoger entre dichos modos.

Esta acción, siguiendo la maqueta definida en la figura \ref{fig:ui_design}, se realizará interactuando con
un botón.

\subsubsection*{ -- Ejecución en un momento determinado}
La interacción con los elementos comentados anteriormente no será efectiva hasta que el usuario indique que
quiere que se realicen, permitiendo así confirmar que los datos introducidos son los correctos.

Esta acción, siguiendo la maqueta definida en la figura \ref{fig:ui_design}, se realizará interactuando con
un botón.

\subsubsection*{ -- Demostración del punto final del brazo}
La interacción con los actuadores definidos anteriormente se verá reflejada en unas pequeñas ventanas
que muestran cómo debería encontrarse el brazo tras realizar los movimientos indicados.

Esta demostración, siguiendo la maqueta definida en la figura \ref{fig:ui_design}, se mostrará mediante
unos dibujos esquemáticos que representan el brazo visto de perfil y desde una vista cenital.

\subsubsection*{ -- Otros requisitos}
Se han considerado operaciones más avanzadas para el control del brazo (como trazar trayectorias o
un control mediante el ratón en un plano 2D) las cuales no se reflejan en este documento ya que se
ha postergado su desarrollo e implementación a una futura versión del sistema.
