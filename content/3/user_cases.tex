\begin{figure}[H]
    \centering
    \includegraphics[width=.7\textwidth]{RS/images/UseCaseDiagram1.png}
    \caption{Diagrama de casos de uso}
    \label{fig:diagrama_casos_uso}
\end{figure}


\begin{table}[H]
    \centering
    \begin{tabularx}{\textwidth}{|c|c|X|}
        \cline{1-3}
        \texttt{0001}                              & \multicolumn{2}{c|}{Encender brazo robótico (\ac{S2})}                                                                                                                      \\ \cline{1-3}
        \textbf{Descripción}                       & \multicolumn{2}{m{13cm}|}{El usuario deberá se capaz de encender el sistema del brazo robótico de manera independiente de la aplicación de control.}
        \\ \cline{1-3}
        \multirow{4}{*}{\textbf{Secuencia Normal}} & \textbf{Paso} & \textbf{Acción}
        \\ \cline{2-3}                    &   1  & El usuario interactúa con el sistema para encenderlo.
        \\ \cline{2-3}                    &   2  & El sistema comprueba que los motores están correctamente conectados.
        \\ \cline{2-3}                    &   3  & Si las comprobaciones son satisfactorias, el sistema continúa con su normal ejecución.
        \\ \cline{1-3}
        \multirow{2}{*}{\textbf{Excepciones}}      & \textbf{Paso}                                                                                                                                        & \textbf{Acción}
        \\ \cline{2-3}                    &   3  & Si las comprobaciones no son satisfactorias el sistema activará un indicador luminoso y se informará del error a \ac{S1}, si está conectado.
        \\ \cline{1-3}
        \textbf{Importancia}                       & \multicolumn{2}{c|}{1}                                                                                                                                                 \\ \cline{1-3}
        \textbf{Comentarios}                       & \multicolumn{2}{c|}{Sin comentarios}                                                                                                                                   \\ \cline{1-3}
    \end{tabularx}
    \caption{Caso de uso \texttt{0001} - Encender brazo robótico (\ac{S2}).}
    \label{tab:CU0001}
    \label{tab:caso_de_uso_encender_brazo_robotico}
\end{table}

\begin{table}[H]
    \centering
    \begin{tabularx}{\textwidth}{|c|c|X|}
        \cline{1-3}
        \texttt{0002}                              & \multicolumn{2}{c|}{Apagar brazo robótico(\ac{S2})}                    \\ \cline{1-3}
        \textbf{Descripción}                       & \multicolumn{2}{m{13cm}|}{El usuario deberá ser capaz de apagar el brazo robótico desconectando la corriente del mismo.}
        \\ \cline{1-3}
        \multirow{4}{*}{\textbf{Secuencia Normal}} & \textbf{Paso}  & \textbf{Acción}
        \\ \cline{2-3}                             &   1            & El usuario interactúa con el sistema para apagarlo.
        \\ \cline{2-3}                             &   2            & El sistema se apaga.
        \\ \cline{1-3}
        \multirow{2}{*}{\textbf{Excepciones}}      & \textbf{Paso}                                                                                                                                        & \textbf{Acción}
        \\ \cline{2-3}                    &     & No existen
        \\ \cline{1-3}
        \textbf{Importancia}                       & \multicolumn{2}{c|}{1}                                                                                                                                                 \\ \cline{1-3}
        \textbf{Comentarios}                       & \multicolumn{2}{c|}{Sin comentarios}                                                                                                                                   \\ \cline{1-3}
    \end{tabularx}
    \caption{Caso de uso \texttt{0002} - Apagar brazo robótico (\ac{S2}).}
    \label{tab:CU0002}
    \label{tab:caso_de_uso_apagar_brazo_robotico}
\end{table}

\begin{table}[H]
    \centering
    \begin{tabularx}{\textwidth}{|c|c|X|}
        \cline{1-3}
        \texttt{0003}        & \multicolumn{2}{c|}{Cerrar aplicación}                                                       
        \\ \cline{1-3}
        \textbf{Descripción} & \multicolumn{2}{m{13cm}|}{El usuario deberá ser capaz de cerrar la aplicación de control de manera independiente al brazo robótico.}
        \\ \cline{1-3}
        \multirow{4}{*}{\textbf{Secuencia Normal}} & \textbf{Paso} & \textbf{Acción}
        \\ \cline{2-3}                    &   1  & El usuario interactúa con la aplicación para cerrarla
        \\ \cline{2-3}                    &   2  & Se comprueba que la  aplicación se puede cerrar de manera segura. Esto implica asegurar que no hay ninguna comunicación en proceso antes de cerrar la apliación
        \\ \cline{2-3}                    &   3  & Se realiza el cierre de la aplicación.
        \\ \cline{1-3}
        \multirow{2}{*}{\textbf{Excepciones}} & \textbf{Paso} & \textbf{Acción}
        \\ \cline{2-3}                        &  2  & La aplicación no se puede cerrar de manera segura.
        \\ \cline{2-3}                        & 2.1 & Se impide el cierre
        \\ \cline{1-3}
        \textbf{Importancia}                 & \multicolumn{2}{c|}{1}           
        \\ \cline{1-3}
        \textbf{Comentarios}                 & \multicolumn{2}{c|}{Sin comentarios}
        \\ \cline{1-3}
    \end{tabularx}
    \caption{Caso de uso \texttt{0003} - Cerrar aplicación.}
    \label{tab:CU0003}
    \label{tab:caso_de_uso_cerrar_aplicación}
\end{table}


\begin{table}[H]
    \centering
    \begin{tabularx}{\textwidth}{|c|c|X|}
        \cline{1-3}
        \texttt{0004}                              & \multicolumn{2}{c|}{Cambiar modo de funcionamiento (\ac{S1})}                                                                                                                      \\ \cline{1-3}
        \textbf{Descripción}                       & \multicolumn{2}{m{13cm}|}{El usuario deberá ser capaz de seleccionar el modo de control del brazo robótico, pudiendo escoger entre control mediante ratón o control mediante parámetros. }
        \\ \cline{1-3}
        \multirow{4}{*}{\textbf{Secuencia Normal}} & \textbf{Paso}                                                                                                                                        & \textbf{Acción}
        \\ \cline{2-3}                    &   1  & El usuario interactúa con la aplicación y selecciona el modo de control del robot.
        \\ \cline{2-3}                    &   2  & El sistema cambia entre modo de control mediante ratón o modo de control mediante parámetros.
        \\ \cline{1-3}
        \multirow{2}{*}{\textbf{Excepciones}}      & \textbf{Paso}                                                                                                                                        & \textbf{Acción}
        \\ \cline{2-3}                    &     &  No existen
        \\ \cline{1-3}
        \textbf{Importancia}                       & \multicolumn{2}{c|}{1}                                                                                                                                                 \\ \cline{1-3}
        \textbf{Comentarios}                       & \multicolumn{2}{c|}{Sin comentarios}                                                                                                                                   \\ \cline{1-3}
    \end{tabularx}
    \caption{Caso de uso \texttt{0004} - Cambiar modo de funcionamiento.}
    \label{tab:CU0004}
    \label{tab:caso_de_uso_cambiar_modo_de_funcionamiento}
\end{table}


\begin{table}[H]
    \centering
    \begin{tabularx}{\textwidth}{|c|c|X|}
        \cline{1-3}
        \texttt{0005}                              & \multicolumn{2}{c|}{Control usando valores numéricos (\ac{S1})}                                                                                                                      \\ \cline{1-3}
        \textbf{Descripción}                       & \multicolumn{2}{m{13cm}|}{El usuario deberá ser capaz de cambiar el valor numérico de cada uno de los parámetros de control del brazo robótico}
        \\ \cline{1-3}
        \multirow{4}{*}{\textbf{Secuencia Normal}} & \textbf{Paso}                                                                                                                                        & \textbf{Acción}
        \\ \cline{2-3}                    &   1  & El usuario interactúa con la aplicación y cambia el valor de los parámetros de control usando el teclado.
        \\ \cline{2-3}                    &   2  & Se comprueba si el valor es correcto y se confirma el cambio del valor numérico.
        \\ \cline{1-3}
        \multirow{2}{*}{\textbf{Excepciones}} & \textbf{Paso}  & \textbf{Acción}
        \\ \cline{2-3}                        &   2  & El valor introducido por el usuario no es correcto y por lo tanto no puede llevarse a la práctica.
        \\ \cline{2-3} 
                                              &  2.1 & Se elimina el valor y se notifica al usuario sobre el error y se le pide que introduzca de nuevo el valor.
        \\ \cline{1-3}
        \textbf{Importancia}                       & \multicolumn{2}{c|}{1}                                                                                                                                                 \\ \cline{1-3}
        \textbf{Comentarios}                       & \multicolumn{2}{c|}{Sin comentarios}                                                                                                                                   \\ \cline{1-3}
    \end{tabularx}
    \caption{Caso de uso \texttt{0005} - Control usando valores numéricos (\ac{S1}).}
    \label{tab:CU0005}
    \label{tab:caso_de_uso_control_usando_valores_numericos}
\end{table}

\begin{table}[H]
    \centering
    \begin{tabularx}{\textwidth}{|c|c|X|}
        \cline{1-3}
        \texttt{0006}        & \multicolumn{2}{c|}{Control describiendo trayectorias}                                      
        \\ \cline{1-3}
        \textbf{Descripción} & \multicolumn{2}{m{13cm}|}{Se permitirá al usuario escoger una trayectoria predefinida que el brazo robótico deberá realizar.}
        \\ \cline{1-3}
        \multirow{4}{*}{\textbf{Secuencia Normal}} & \textbf{Paso} & \textbf{Acción}
        \\ \cline{2-3}                    &   1  & El usuario selecciona una trayectoria a realizar.
        \\ \cline{2-3}                    &   2  & Se realiza dicha trayectoria
        \\ \cline{1-3}
        \multirow{2}{*}{\textbf{Excepciones}} & \textbf{Paso} & \textbf{Acción}
        \\ \cline{2-3}                    &      &  No existen
        \\ \cline{1-3}
        \textbf{Importancia}                 & \multicolumn{2}{c|}{1}           
        \\ \cline{1-3}
        \textbf{Comentarios}                 & \multicolumn{2}{m{13cm}|}{\textbf{Esta característica no se implementa ya que se posterga para una futura versión.}}
        \\ \cline{1-3}
    \end{tabularx}
    \caption{Caso de uso \texttt{0006} - Control describiendo trayectorias.}
    \label{tab:CU0006}
    \label{tab:caso_de_uso_control_describiendo_trayectorias}
\end{table}

\begin{table}[H]
    \centering
    \begin{tabularx}{\textwidth}{|c|c|X|}
        \cline{1-3}
        \texttt{0007}        & \multicolumn{2}{c|}{Control usando controles gráficos}                                      
        \\ \cline{1-3}
        \textbf{Descripción} & \multicolumn{2}{m{13cm}|}{La interfaz gráfica de la aplicación debe ofrecer control sobre los parámetros del brazo robótico mediante \textit{sliders}.}
        \\ \cline{1-3}
        \multirow{4}{*}{\textbf{Secuencia Normal}} & \textbf{Paso} & \textbf{Acción}
        \\ \cline{2-3}                    &   1  & El usuario interactúa con la aplicación y mueve los \textit{sliders} para variar los parámetros del brazo robótico.
        \\ \cline{2-3}                    &   2  & Se verifica si se puede realizar dicho movimiento y se ejecuta el cambio en los parámetros.
        \\ \cline{1-3}
        \multirow{2}{*}{\textbf{Excepciones}} & \textbf{Paso} & \textbf{Acción}
        \\ \cline{2-3}                    &   2   &  La posición no es alcanzable o no se puede llevar a la práctica.
        \\ \cline{2-3}                    &  2.1  &  Se notifica el error.
        \\ \cline{1-3}
        \textbf{Importancia}                 & \multicolumn{2}{c|}{1}           
        \\ \cline{1-3}
        \textbf{Comentarios}                 & \multicolumn{2}{c|}{Sin comentarios}
        \\ \cline{1-3}
    \end{tabularx}
    \caption{Caso de uso \texttt{0007} - Control usando controles gráficos.}
    \label{tab:CU0007}
    \label{tab:caso_de_uso_control_usando_controles_graficos}
\end{table}

\begin{table}[H]
    \centering
    \begin{tabularx}{\textwidth}{|c|c|X|}
        \cline{1-3}
        \texttt{0008}        & \multicolumn{2}{c|}{Control usando ratón}                                      
        \\ \cline{1-3}
        \textbf{Descripción} & \multicolumn{2}{m{13cm}|}{Se permitirá al usuario controlar el brazo robótico de manera directa con el movimiento del ratón}
        \\ \cline{1-3}
        \multirow{4}{*}{\textbf{Secuencia Normal}} & \textbf{Paso} & \textbf{Acción}
        \\ \cline{2-3}                    &   1  & El usuario mueve el ratón realizando movimientos libres.
        \\ \cline{2-3}                    &   2  & Se comprueba que el movimiento no se sale de los margenes permitidos
        \\ \cline{2-3}                    &   3  & Se realiza el movimiento
        \\ \cline{1-3}
        \multirow{2}{*}{\textbf{Excepciones}} & \textbf{Paso} & \textbf{Acción}
        \\ \cline{2-3}                    &   2   &  Si los movimientos se salen de los margenes permitidos no se realizan.
        \\ \cline{1-3}
        \textbf{Importancia}                 & \multicolumn{2}{c|}{1}           
        \\ \cline{1-3}
        \textbf{Comentarios}                 & \multicolumn{2}{m{13cm}|}{\textbf{Esta característica no se implementa ya que se posterga para una futura versión.}}
        \\ \cline{1-3}
    \end{tabularx}
    \caption{Caso de uso \texttt{0008} - Control usando ratón.}
    \label{tab:CU0008}
    \label{tab:caso_de_uso_control_usando_ratón}
\end{table}

\begin{table}[H]
    \centering
    \begin{tabularx}{\textwidth}{|c|c|X|}
        \cline{1-3}
        \texttt{0009}        & \multicolumn{2}{c|}{Accionar el \textit{end--effector}}                                      
        \\ \cline{1-3}
        \textbf{Descripción} & \multicolumn{2}{m{13cm}|}{Se permite al usuario abrir y cerrar la pinza}
        \\ \cline{1-3}
        \multirow{4}{*}{\textbf{Secuencia Normal}} & \textbf{Paso} & \textbf{Acción}
        \\ \cline{2-3}                    &   1  & El usuario interactúa con la aplicación para abrir y cerrar el \textit{end--effector}
        \\ \cline{2-3}                    &   2  & Se cambia el estado del \textit{end--effector} según sea necesario. 
        \\ \cline{1-3}
        \multirow{2}{*}{\textbf{Excepciones}} & \textbf{Paso} & \textbf{Acción}
        \\ \cline{2-3}                    &      &  No existe
        \\ \cline{1-3}
        \textbf{Importancia}                 & \multicolumn{2}{c|}{1}           
        \\ \cline{1-3}
        \textbf{Comentarios}                 & \multicolumn{2}{m{13cm}|}{\textbf{Esta característica no se implementa ya que se posterga para una futura versión.}}
        \\ \cline{1-3}
    \end{tabularx}
    \caption{Caso de uso \texttt{0009} - Accionar el \textit{end--effector}.}
    \label{tab:CU0009}
    \label{tab:caso_de_uso_accionar_end_effector}
\end{table}

\begin{table}[H]
    \centering
    \begin{tabularx}{\textwidth}{|c|c|X|}
        \cline{1-3}
        \texttt{0010}        & \multicolumn{2}{c|}{Activar modo \textit{debug}}                                      
        \\ \cline{1-3}
        \textbf{Descripción} & \multicolumn{2}{m{13cm}|}{Se permite al usuario activar un modo tal que se pueda mandar al \ac{S2} el código de control del brazo robótico}
        \\ \cline{1-3}
        \multirow{4}{*}{\textbf{Secuencia Normal}} & \textbf{Paso} & \textbf{Acción}
        \\ \cline{2-3}                    &   1  & El usuario interactúa con S2 para ponerlo en modo debug.
        \\ \cline{2-3}                    &   2  & El sistema comprueba que el cambio de modo se puede hacer de manera segura. Es decir, no hay una comuniación en proceso antes de realizar el cambio de modo.
        \\ \cline{2-3}                    &   3  & El sistema cambia de modo.
        \\ \cline{1-3}
        \multirow{2}{*}{\textbf{Excepciones}} & \textbf{Paso} & \textbf{Acción}
        \\ \cline{2-3}                    &   2  & El sistema detecta que el cambio de modo no se puede hacer de manera segura e impide que este se realice. Se informará del error a \ac{S1}, si está conectado.
        \\ \cline{1-3}
        \textbf{Importancia}                 & \multicolumn{2}{c|}{1}           
        \\ \cline{1-3}
        \textbf{Comentarios}                 & \multicolumn{2}{m{13cm}|}{\textbf{Esta característica no se implementa ya que se posterga para una futura versión.}}
        \\ \cline{1-3}
    \end{tabularx}
    \caption{Caso de uso \texttt{0010} - Activar el modo \textit{debug}.}
    \label{tab:CU0010}
    \label{tab:caso_de_uso_activar_modo_debug}
\end{table}

\begin{table}[H]
    \centering
    \begin{tabularx}{\textwidth}{|c|c|X|}
        \cline{1-3}
        \texttt{0011}        & \multicolumn{2}{c|}{Desactivar modo \textit{debug}}                                      
        \\ \cline{1-3}
        \textbf{Descripción} & \multicolumn{2}{m{13cm}|}{Se permite al usuario desactivar el modo debug tal que sea posible emplear el sistema de manera normal}
        \\ \cline{1-3}
        \multirow{4}{*}{\textbf{Secuencia Normal}} & \textbf{Paso} & \textbf{Acción}
        \\ \cline{2-3}                    &   1  & El usuario interactúa con \ac{S2} para desactivar el modo debug.
        \\ \cline{2-3}                    &   2  & El sistema comprueba que el cambio de modo se puede hacer de manera segura.
        \\ \cline{2-3}                    &   3  & El sistema cambia de modo.
        \\ \cline{1-3}
        \multirow{2}{*}{\textbf{Excepciones}} & \textbf{Paso} & \textbf{Acción}
        \\ \cline{2-3}                    &   2  & El sistema detecta que el cambio de modo no se puede hacer de manera segura e impide que este se realice. Se informara del error a \ac{S1}, si esta conectado.
        \\ \cline{1-3}
        \textbf{Importancia}                 & \multicolumn{2}{c|}{1}           
        \\ \cline{1-3}
        \textbf{Comentarios}                 & \multicolumn{2}{m{13cm}|}{\textbf{Esta característica no se implementa ya que se posterga para una futura versión.}}
        \\ \cline{1-3}
    \end{tabularx}
    \caption{Caso de uso \texttt{0011} - Desactivar el modo \textit{debug}.}
    \label{tab:CU0011}
    \label{tab:caso_de_uso_desactivar_modo_debug}
\end{table}