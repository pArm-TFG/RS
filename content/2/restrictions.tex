% Dado que actualmente no se presenta ninguna limitación de presupuesto, se busca en el proyecto intentar reducir los costes todo lo posible, para permitir a cualquiera poder acceder a los recursos que se necesitan para desarrollar el proyecto. Por ello, se propone usar el dsPIC33EP512GM604 y montar la placa con los componentes necesarios.
Por estar ya disponible y por cubrir correctamente las necesidades de los periféricos
necesarios para desarrollar este proyecto, se recomienda como alternativa inicial para
el sistema \ac{S2} utilizar el dsPIC33EP512GM604.

En cualquier caso, como se ha mencionado anteriormente, es necesario que:

\begin{itemize}
    \item Se provea de una interfaz para la comunicación que permita comunicarse con el sistema \ac{S1} de manera simultánea y con alta capacidad.
    \item El sistema ha de consumir la menor energía posible, entrando en el modo de \textit{deep--sleep} cuando fuera posible.
    \item La estructura de \ac{S2} ha de ser imprimible en 3D, permitiendo así replicarlo.
    \item El sistema \ac{S1} ha de poder ejecutar aplicaciones Python según lo propuesto anteriormente, en particular, la versión de este superior a la 3.6. En otro caso, el sistema habrá de poder ejecutar la aplicación diseñada sin problemas e indiferentemente del sistema operativo.
    \item Todo lo realizado en el proyecto ha de ser \ac{OS} y \ac{OH}, permitiendo así que cualquiera pueda acceder y estudiar el proyecto.
\end{itemize}