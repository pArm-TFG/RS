Dado que actualmente no se presenta ninguna limitación de presupuesto, se busca en el proyecto intentar reducir los costes todo lo posible, para permitir a cualquiera que pueda acceder a los recursos que se necesitan para desarrollar el proyecto. Por ello, se propone usar si es posible el PIC16F18326-I/P y montar la placa con los componentes necesarios.

En cualquier caso, como se ha mencionado anteriormente, es necesario que:

\begin{itemize}
    \item Se provea de una interfaz UART para comunicación por puerto serie \ac{USB} para comunicarse con \ac{S1}.
    \item El sistema ha de consumir la menor energía posible, entrando en el modo de \textit{deep--sleep} cuando fuera posible.
    \item La estructura de \ac{S2} ha de ser imprimible en 3D, permitiendo así replicarlo.
    \item El sistema \ac{S1} ha de poder ejecutar aplicaciones Python, siendo la versión de este superior a la 3.6.
    \item Además, el sistema \ac{S1} ha de tener capacidad de cómputo suficiente para realizar cálculos matriciales de forma efectiva. Para ello, se sugiere que el equipo disponga al menos de un procesador con dos núcleos y 512 MB de memoria RAM.
    \item Todo lo realizado en el proyecto ha de ser \ac{OS} y \ac{OH}, permitiendo así que cualquiera pueda acceder y estudiar el proyecto.
\end{itemize}