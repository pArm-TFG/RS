El \pArm{} se basa en el trabajo inicial del $\mu$Arm, no utilizando directamente lo desarrollado por la empresa UFACTORY sino aprovechando el trabajo ya realizado y los recursos disponibles para estudiarlos. Si bien es cierto que el $\mu$Arm ya es un sistema avanzado y capaz, como se explicó en la Introducción (\ref{ch:intro}), se pretende estudiar y desarrollar un sistema propio el cual pueda servir para ayudar y facilitar la entrada a este tipo de tecnologías a otras personas, haciéndolo comprensible y, aprovechando la tecnología de la impresión en 3D, fabricable por uno mismo.

Además, en relación a los \ac{ODS}, con el desarrollo de este sistema se pretende trabajar en:

\begin{itemize}
    \item [4 -] Educación de Calidad\footnote{\url{https://www.un.org/sustainabledevelopment/es/education/}}.
    \item [7 -] Energía Asequible y No Contaminante\footnote{\url{https://www.un.org/sustainabledevelopment/es/energy/}}.
    \item [10 -] Reducción de las desigualdades\footnote{\url{https://www.un.org/sustainabledevelopment/es/inequality/}}.
\end{itemize}

Para el primero, se tiene en cuenta que el producto se desarrollará siguiendo las iniciativas \ac{OS} y \ac{OH}, las cuales facilitan el acceso a la información a cualquiera que la requiera. Además, se facilitará el desarrollo al completo detallado y explicado, con la resolución de los problemas pertinentes y el porqué de ella.

Para el segundo, el \pArm{} utilizará la electricidad como fuente de energía, evitando así otras más contaminantes como las producidas por combustibles fósiles. En añadido, se trabajará para que el consumo de energía sea el menor posible, permitiendo así un mayor tiempo de uso con la misma fuente de alimentación y no abusando de los recursos de los que se disponen.

Finalmente, se pretende hacer que el \pArm{} tenga un coste bajo, permitiendo así el acceso a los recursos y los procesos de fabricación a todo el mundo que pudiera estar interesado y que disponga de la cantidad mínima necesaria para poder poner en funcionamiento el brazo robótico.

Por otra parte, el \pArm{} es dependiente de otro sistema que lo controle, ya que no se plantea como sistema autónomo. Por consiguiente, se proponen diversos métodos de conexión entre el brazo y dicho sistema. Por ejemplo, se puede utilizar el puerto serie (\ac{USB}) o bien comunicaciones inalámbricas, como \textit{Bluetooth} y \textit{WiFi}. Además, debido a su disponibilidad multiplataforma, se propone el uso de Python como aplicación de control. 

De ahora en adelante, se denominará ``\ac{S1}'' al equipo que controla al \pArm{}; y ``\ac{S2}'' al brazo robótico en sí.

Para este proyecto, se ha de desarrollar el \ac{SW} que se ejecutará en el \ac{S1} y tanto el \ac{SW} como el \ac{HW} que irán en el \ac{S2}, así como la estructura del mismo.

\subsection{Interfaz del sistema}
En un principio, el sistema estará dividido en dos módulos:

\subsection*{\ac{S1}}
El \ac{S1} consiste en un equipo el cual controlará el brazo robótico (\ac{S2}). Para ello, tal y como se planteó anteriormente, se propone como lenguaje de programación Python, el cual soporta la ejecución con \ac{GUI}.

En lo referente al sistema operativo, al ser una aplicación en Python la cual es multiplataforma, no se define ninguna restricción respecto al mismo.

Finalmente, se plantea la conexión con el \ac{S2} utilizando el puerto serie \ac{USB}, por lo que también será necesario que el equipo anfitrión \ac{S1} disponga de una conexión de ese estilo.

En resumen (ver la tabla \ref{tab:s1_requirements}):

\begin{table}[H]
    \centering
    \begin{tabularx}{\textwidth}{| c | X | X |}
        \hline
        \textbf{Componente} & \textbf{Función} & \textbf{Restricciones} \\
        \hline\hline
        Sistema Operativo & Hospedar y ejecutar la aplicación Python que controlará el brazo robótico. & Debe poder ejecutar aplicaciones Python con \ac{GUI}, por ejemplo, \ac{GTK}. \\
        \hline
        \textit{Conexión con \ac{S2}} & Permitir la comunicación con el sistema \ac{S2} en modo dúplex. & Velocidad adaptable (\textit{baud--rate)} y capacidad para gran ancho de banda. \\
        \hline
        Python & Control del sistema \ac{S2} y monitorización del estado del mismo. & Versión Python $\geqslant$ 3.6.* \\
        \hline
    \end{tabularx}
    \caption{Requisitos del sistema \ac{S1}.}
    \label{tab:s1_requirements}
\end{table}

En principio, no será necesaria la conexión a Internet, pero tampoco se descarta el uso de la misma en el proyecto a la hora de poder recibir actualizaciones o en lo referente a futuras mejoras.

En añadido, pese a que no se restringe, se sugiere que el equipo que hospede la aplicación disponga de un procesador multinúcleo así como suficiente memoria \ac{RAM} para poder manejarlas.

\subsection*{\ac{S2}}
Para el sistema \ac{S2} no se ha pensado en ningún microprocesador ni \ac{SoC} en particular, pero se han contemplado algunos que cumplen con las características requeridas (ver la tabla \ref{tab:s2_chips}).

Será necesario que el circuito escogido disponga de algún tipo de entrada de las propuestas para la comunicación con el \ac{S1}. Debido a la característica descrita en la tabla \ref{tab:s1_requirements} sobre la interfaz de comunicación, no será estrictamente necesario que la velocidad sea adaptable (ya que se asume que se adaptará en el \ac{S1}); sin embargo, sí será requisito fundamental que la conexión sea dúplex y que soporte gran cantidad de datos con las menores pérdidas posibles.

Por otra parte, el microcontrolador deberá poder modular señales \ac{PWM} para controlar los distintos motores de los que dispondrá el brazo. Sin embargo, en caso de que finalmente el \textit{chip} escogido no disponga de dicha modularización, se podrán usar motores los cuales cuenten con un \textit{driver} que permitan controlarlos usando señales digitales y/o analógicas.

Finalmente, a raíz del punto anterior, será imprescindible que el sistema escogido tenga la capacidad de controlar señales digitales y analógicas, permitiendo así mayor versatilidad en el desarrollo del producto final.

Teniendo en cuenta lo anterior, se plantea el uso de los siguientes dispositivos (ver tabla \ref{tab:s2_chips}):

\begin{table}[H]
    \centering
    \begin{tabularx}{\textwidth}{| c | X | X | X |}
        \hline
        \textbf{Placa} & \textbf{Ventajas} & \textbf{Desventajas} & \textbf{ID \textit{RS--Online} y precio} \\
        \hline
        ESP8266 & \ac{SoC} bastante barato (5\EUR{}) con conexión WiFi y modo de bajo consumo & Señal \ac{PWM} generada por \ac{SW}; poca cantidad de \ac{GPIO} (6). & 124-5505 -- 19,29\EUR{} \\
        \hline
        ESP32 & \ac{SoC} con procesador de dos núcleos que permite comunicaciones WiFi y Bluetooth & No cuenta con \ac{GPIO} pero permite la comunicación mediante el protocolo I²C. & 188-5441 -- 25,29\EUR{} \\
        \hline
        PIC16F18326-I/P & Microcontrolador de 8 bits de baja potencia de consumo y bajo precio con capacidad de modularizar hasta dos señales \ac{PWM} y con más memoria RAM que otros componentes de su familia. Finalmente, cuenta con bastantes salidas \ac{GPIO}, suficientes como para añadir más componentes al sistema. & No está integrada en una placa (\ac{SoC}) por lo que habría que hacer toda la lógica del diseño \ac{HW}. No dispone de conexiones de red (aunque no son necesarias) y la capacidad de cómputo, en comparación con las otras propuestas, es menor. & 124-1554 -- 1,375\EUR{} \\
        \hline
    \end{tabularx}
    \caption{Posibles \textit{chips} que se han planteado para el proyecto.}
    \label{tab:s2_chips}
\end{table}


\subsection{Interfaz de usuario}

El usuario final del producto solamente interactuara de manera directa con el \ac{S1}.
Para que esta interacción sea posible, se desarrollara un panel de control que permita al usuario definir movimientos que el robot deberá realizar. El panel de control se mostrará en una sola pantalla y permitirá al usuario, mediante una interfaz gráfica sencilla, mover de manera independiente cada uno de los motores del robot, o bien mediante el uso del ratón, describir trayectorias que el robot realizara en tiempo real replicando el movimiento del ratón.

\subsection{Interfaz \textit{hardware}}

El \ac{S2} deberá de tener una interfaz con el \ac{HW} tal que se puedan rotar los motores que controlan el movimiento del robot. Por tanto, el microcontrolador que sea elegido para el proyecto final deberá permitir dicho control en base a las características de los motores. Por otro lado, si los motores devuelven su posición, el microcontrolador deberá ser capaz de recibir este dato para poder enviarlo al \ac{S1}.

\subsection{Interfaz de comunicaciones}
Debido a la naturaleza del proyecto deberá existir una interfaz de comunicaciones que permita el envío y recepción de datos desde el \ac{S1} al \ac{S2} y viceversa. Según lo indicado en el apartado Descripción General (\ref{Descripción general}), esta comunicación se propone hacer mediante \ac{UART}, \textit{Bluetooth}, \textit{WiFi}, o algún estándar de comunicación que permita una conexión bidireccional simultanea y por tanto permite controlar el robot a la vez que se recibe información referente al estado de este. 

Además con este estándar la comunicación es serie y cumple con las limitaciones de diseño que se han impuesto en la recepción de datos en el \ac{S2}.

\subsection{Memoria}

\begin{itemize}

    \item En cuanto a la memoria de programa se refiere, el microcontrolador del sistema \ac{S2} debe tener una memoria suficientemente grande como para poder albergar el \ac{SW} que recibirá datos del \ac{S1} y que actuará sobre los motores en base a estos.
    
    \item Por otro lado, la memoria \ac{RAM} deberá ser tal que permita realizar las operaciones necesarias para ejecutar los movimientos. En caso de que se considere posible la completa implementación de la lógica en el sistema \ac{S2}, el microcontrolador deberá tener memoria principal suficiente como para conseguir realizar los cálculos matriciales relacionados con los distintos movimientos.
    
\end{itemize}
Por otro lado, la memoria principal deberá ser tal que permita las operaciones necesarias para realizar los movimientos en tiempo de ejecución. Cabe destacar que una de las prioridades principales del equipo es ser efectivos en cuanto a la ocupación de memoria del código que desarrolla. Por ello, se buscará reducir el tamaño de este al mínimo manteniendo las funcionalidades necesarias para cumplir los objetivos establecidos.   

\subsection{Operaciones}

Los usuarios deberán desempeñar acciones tales que generen los movimientos deseados en el brazo. Estas acciones pueden implicar interactuar con los \textit{widgets} presentes en el panel de control o bien efectuar movimientos con el ratón para que el robot los desempeñe directamente.
