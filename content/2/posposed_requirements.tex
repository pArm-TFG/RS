En esta sección se describen algunos requisitos del sistema que se postergan a futuras implementaciones o versiones del proyecto.

En el comienzo del proyecto se plantearon algunas funcionalidades y requisitos que, finalmente, se han decidido postergar a futuras implementaciones del proyecto, principalmente debido a su complejidad. En la siguiente lista se presentan las mas relevantes, las cuales representan posibles mejoras futuras del \pArm{}:
\begin{itemize}
    \item Implementación del sistema de control y planificación de trayectorias en el microcontrolador del \pArm{}, de esta forma se busca centralizar el computo en \ac{S2}.
    \item Implementación de un sistema de descripción de trayectorias mediante imitación de movimientos realizados por el usuario, es decir, el usuario podría mover físicamente el \pArm{} y memorizaría dicha trayectoria para posteriormente describirla.
    \item Construcción e implementación de diversos tipos de \textit{end--effector} para el \pArm{}, los cuales le dotarían de nuevas funcionalidades en cuanto a manejar objetos.
    \item Implementación de las estructura física del \pArm{} utilizando materiales metálicos para mejorar su resistencia y estabilidad. Junto con esta mejora, se podrían utilizar nuevos rotores para dotar al \pArm{} de una mayor capacidad de carga.
\end{itemize}

  