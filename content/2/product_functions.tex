Las funcionalidades principales del brazo robótico han sido descritas de forma introductoria en apartados anteriores de este documento.

En general, existen dos funcionalidades principales que caracterizan tanto al \pArm{} como al sistema de control del mismo:
\begin{itemize}
    \item La funcionalidad principal del brazo robótico \ac{S2} es la de realizar movimientos dentro de su campo de movimiento y describir trayectorias previamente planificadas o calculadas en el momento. Mediante este movimiento, se pretende transportar objetos de poco peso.
    
    Además, para agilizar el funcionamiento y el procesado de las órdenes, será el sistema \ac{S2} el que gestione, compute y realice los movimientos que recibe por parte de \ac{S1}, quedando este último para la interacción con el usuario y la gestión de \ac{S2}.
    
    \item El sistema de control \ac{S1} ofrece la funcionalidad principal de planificar trayectorias y controlar el movimiento del brazo. Este sistema se muestra al usuario mediante una interfaz gráfica, la  cual permite al usuario controlar el movimiento del brazo mediante la modificación de diversos parámetros.
\end{itemize}
