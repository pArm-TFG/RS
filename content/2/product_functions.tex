Las funcionalidades principales del brazo robótico han sido descritas de forma introductoria en apartados anteriores de este documento.

En general, existen dos funcionalidades principales que caracterizan tanto al \pArm{} como al sistema de control del mismo:
\begin{itemize}
    \item La funcionalidad principal del brazo robótico \ac{S2} es la de realizar movimientos dentro de su campo de movimiento y describir trayectorias previamente planificadas o calculadas en el momento. Mediante este movimiento, se pretende transportar objetos de poco peso.
    
    Cabe destacar que el brazo robótico \ac{S2} recibe ordenes del \ac{S1} y procesa las mismas utilizando el microcontrolador que posee, por lo tanto el computo principal se realiza en \ac{S1}.
    
    \item El sistema de control en  del brazo robótico \ac{S2} ofrece la funcionalidad principal de planificar trayectorias y controlar el movimiento del brazo. Este sistema se muestra al usuario mediante una interfaz gráfica en \ac{S1}, la  cual permite al usuario controlar el movimiento del brazo mediante la modificación de sus parámetros.
\end{itemize}
