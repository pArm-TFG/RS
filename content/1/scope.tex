
El objetivo principal de este proyecto fin de grado es construir una brazo robótico similar al manipulador $\mu$Arm, al cual se le ha asignado el nombre \ac{pArm}.

Este brazo robótico debe ser capaz de moverse libremente dentro de su campo de movimiento, el cual está limitado por su estructura física. Además, el \pArm{} debe ser capaz de coger, transportar y depositar objetos de poco peso y, en consecuencia, debe ser capaz de describir trayectorias previamente planificadas o calculadas en el momento.

Es importante destacar que, dado que el brazo robótico \pArm{} no está sensorizado, este no será capaz de moverse de forma completamente autónoma ni de imitar movimientos realizados por el usuario.

Cabe destacar que el brazo robótico está controlado mediante un microcontrolador. Sin embargo, las instrucciones de movimiento y trayectorias no se computan, en principio, en el mismo sino en un ordenador auxiliar.

Debido a la estructura física, tamaño y materiales de fabricación, el \pArm{} no es un brazo robótico pensado para la realización de tareas industriales ni para el transporte de cargas pesadas.

En relación a lo anteriormente mencionado, la aplicación principal del \pArm{} es didáctica, dado que se busca construir un brazo robótico económico y sencillo que facilite la introducción de los usuarios a este tipo de tecnologías.
