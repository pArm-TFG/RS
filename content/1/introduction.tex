El $\mu$Arm es un brazo robótico creado por la compañía UFACTORY\footnote{\url{https://www.ufactory.cc/\#/en/uarmswift}} el cual se ha diseñado con propósito principalmente didáctico.

En la actualidad, se puede obtener uno a través de su página web o de proveedores externos, pero no está previsto fabricar más, por lo que en un tiempo estará fuera de existencias.

Debido a su propósito didáctico, todos los recursos sobre el manipulador son de código libre, por lo que resultan accesibles a cualquiera que los necesite. Entre otros, se encuentran\footnote{todos los elementos descritos se encuentran disponibles tanto en \href{https://github.com/uArm-Developer}{GitHub} como en la web de \href{https://www.ufactory.cc/\#/en/support/download/pro}{UFACTORY}}:

\begin{itemize}
    \item \textit{Firmware} del $\mu$Arm Swift Pro.
    \item \ac{SDK} de Python para el $\mu$Arm Swift Pro.
    \item \textit{Firmware} que maneja el controlador del brazo.
    \item \ac{ROS} para el $\mu$Arm Swift Pro.
    \item Distintos ejemplos para toda la gama de brazos robóticos.
    \item \textit{$\mu$Arm Creator Studio}.
    \item Visión esquemática de las conexiones de la placa Arduino.
    \item Modelos 3D del brazo robótico.
    \item Guías de usuario, desarrollador y especificaciones técnicas.
\end{itemize}

Aprovechando dichos recursos, se pretende desarrollar un brazo robótico basado en el $\mu$Arm que esté impreso en 3D y sea controlado por un microcontrolador en conjunción con un ordenador cualquiera. Aprovechando los recursos provistos por UFACTORY, se busca que el brazo desarrollado sea más barato de construir (frente a los casi 800\euro{} que cuesta el original) y que pueda ser desarrollado por cualquiera con acceso a Internet y a los recursos necesarios, a saber, una impresora en 3D y un \ac{SW} de impresión en 3D.